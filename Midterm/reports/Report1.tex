\documentclass[11pt,letterpaper]{article}
\usepackage[utf8]{inputenc}
\usepackage[margin=1in]{geometry}
\usepackage{amsmath}
\usepackage{amssymb}
\usepackage{graphicx}
\usepackage{booktabs}
\usepackage{longtable}
\usepackage{array}
\usepackage{multirow}
\usepackage{hyperref}
\usepackage{fancyhdr}
\usepackage{titlesec}
\usepackage{setspace}
\usepackage[table]{xcolor}
\usepackage{float}
\usepackage{placeins}
\usepackage{listings}
\usepackage{xcolor}

% SAS code formatting
\lstdefinestyle{sasstyle}{
    language=SAS,
    basicstyle=\ttfamily\footnotesize,
    keywordstyle=\color{blue}\bfseries,
    commentstyle=\color{gray}\itshape,
    stringstyle=\color{red},
    numbers=left,
    numberstyle=\tiny\color{gray},
    stepnumber=1,
    numbersep=8pt,
    backgroundcolor=\color{white},
    showspaces=false,
    showstringspaces=false,
    showtabs=false,
    frame=single,
    rulecolor=\color{black},
    tabsize=2,
    captionpos=b,
    breaklines=true,
    breakatwhitespace=true,
    postbreak=\mbox{\textcolor{gray}{$\hookrightarrow$}\space},
    escapeinside={(*@}{@*)},
    morekeywords={proc, data, set, run, model, class, strata, time, assess, PH, resample},
    % Frame settings for page breaks - add lines at top and bottom when breaking
    % framexleftmargin=5pt,
    % framexrightmargin=5pt,
    % framextopmargin=5pt,
    % framexbottommargin=5pt
}

% Page style
\pagestyle{fancy}
\fancyhf{}
\rhead{P8110 Midterm Project - Group 6}
\lhead{Applied Regression II}
\rfoot{Page \thepage}
\setlength{\headheight}{14pt}
\setlength{\footskip}{40pt}  % Match footer distance to header distance
\renewcommand{\headrulewidth}{0.4pt}  % Header line (default)
\renewcommand{\footrulewidth}{0.4pt}  % Footer line (add this)

% Table caption spacing
\usepackage{caption}
% Slightly tighter caption spacing
\captionsetup[table]{skip=10pt}
\captionsetup[figure]{skip=10pt}
% Slightly smaller caption fonts for compactness
\captionsetup{font=small,labelfont=bf}

% Title formatting
\titleformat{\section}{\LARGE\bfseries}{\thesection}{1em}{}[\vspace{0.2em}]
\titleformat{\subsection}{\Large\bfseries}{\thesubsection}{1em}{}[\vspace{0.15em}]
\titleformat{\subsubsection}{\large\bfseries}{\thesubsubsection}{1em}{}[\vspace{0.12em}]

% Paragraph spacing: tighter paragraphs with a small indent
\setlength{\parskip}{0.2em}
\setlength{\parindent}{0.8em}

% Tighter section spacing
\titlespacing*{\section}{0pt}{2.0ex plus 0.4ex minus 0.3ex}{1.0ex}
\titlespacing*{\subsection}{0pt}{1.5ex plus 0.3ex minus 0.2ex}{0.8ex}
\titlespacing*{\subsubsection}{0pt}{1.1ex plus 0.2ex minus 0.2ex}{0.7ex}

% Tighter float spacing
\setlength{\floatsep}{10pt}
\setlength{\textfloatsep}{12pt}
\setlength{\intextsep}{10pt}
% Allow more floats on a page and higher float/text ratios
\setcounter{topnumber}{2}
\setcounter{bottomnumber}{2}
\setcounter{totalnumber}{4}
\renewcommand{\topfraction}{0.92}
\renewcommand{\bottomfraction}{0.85}
\renewcommand{\textfraction}{0.08}
\renewcommand{\floatpagefraction}{0.9}

% Tighter tables
\setlength{\tabcolsep}{5pt}
\renewcommand{\arraystretch}{1.02}
% Slightly tighten booktabs rules
\setlength{\aboverulesep}{0.3ex}
\setlength{\belowrulesep}{0.3ex}
\setlength{\abovetopsep}{0.3ex}
\setlength{\belowbottomsep}{0.3ex}

% Tighter lists
\usepackage{enumitem}
\setlist{itemsep=0.15em, topsep=0.2em, parsep=0pt}

% Hyperref setup
\hypersetup{
    colorlinks=true,
    linkcolor=blue,
    filecolor=magenta,
    urlcolor=cyan,
    citecolor=blue
}

\begin{document}

% Title Page
\begin{titlepage}
    \centering
    \vspace*{1.2cm}

    {\LARGE\bfseries P8110 Applied Regression II\par}
    \vspace{0.5cm}
    {\Large Midterm Project Report 1\par}
    \vspace{1.2cm}

    {\Large\bfseries Survival Analysis of Depression Onset and Substance Abuse in Offspring of Depressed Parents\par}
    \vspace{1.8cm}

    {\large Group 6\par}
    \vspace{0.5cm}
    {\large
    Leah Li (YL5828)\\
    Xuange Liang (XL3493)\\
    Zexuan Yan (ZY2654)\\
    Yiwen Zhang (YZ4994)\\
    }
    \vspace{2cm}

    {\large \today\par}

    \vfill
\end{titlepage}

% Table of Contents
\tableofcontents
% Avoid forcing a new page here to keep sections continuous

\newpage
% Main Content
\section{Data Preparation}

\subsection{Survival Outcome Definitions}

\noindent\textbf{Depression Onset (Pre-pubertal, age $<13$):}
\begin{itemize}
    \item All 220 subjects included in analysis
    \item Start time: Birth (age = 0)
    \item Event: Depression onset before age 13 (\texttt{BEDEPON} $<$ 13)
    \item Censoring: All others censored at age 13 (either had depression at/after 13, or never had depression)
    \item This creates a restricted follow-up period (0--13 years) to focus on pre-pubertal onset
\end{itemize}

\noindent\textbf{Depression Onset (Adolescent/early adulthood, age $\geq 13$):}
\begin{itemize}
    \item Includes only subjects who reached age 13 without depression (left-truncated at age 13)
    \item Excludes: (1) those with depression onset $<$13 years, and (2) those censored before age 13
    \item Start time: Age 13
    \item Event: Depression onset at age $\geq$13 (\texttt{BEDEPON} $\geq$ 13)
    \item Censoring: No depression by interview age (\texttt{PTAGE})
    \item This is a conditional analysis among those ``at risk'' at age 13
\end{itemize}

\noindent\textbf{Substance Abuse Onset:}
\begin{itemize}
    \item Start time: Birth (age = 0); End time: Age of substance abuse onset (\texttt{BESUBON}), censored at \texttt{PTAGE} if no event
\end{itemize}

\subsection{Data Cleaning}

Logical consistency checks were performed on all variables:
\begin{itemize}
    \item Records with status=1 (diagnosed) but missing onset age (-1) were removed as incomplete
    \item Records with onset age present but status=0 were corrected to status=1 for consistency
    \item This procedure was applied to both depression (\texttt{DSMDEPHR}, \texttt{BEDEPON}) and substance abuse variables (\texttt{DSMSUBHR}, \texttt{BESUBON})
\end{itemize}

\textit{Note: Complete SAS code for data preparation is provided in Appendix~A.1.}

\subsection{Descriptive Statistic Table}

\begin{table}[H]
\centering
\caption{Demographic and Clinical Characteristics of the Offspring Stratified by Parental Depression Status}
\label{tab:descriptive}
\begin{tabular}{lcc}
\toprule
\textbf{Variable} & \textbf{Depressed Index Parent} & \textbf{Normal Index Parent} \\
& \textbf{Group (N=125)} & \textbf{Group (N=95)} \\
& \textbf{n (\%)} & \textbf{n (\%)} \\
\midrule
\textbf{Sex of Child} & & \\
\quad Male & 63 (50.4\%) & 42 (44.2\%) \\
\quad Female & 62 (49.6\%) & 53 (55.8\%) \\
\addlinespace
\textbf{Child depression status} & & \\
\quad Ever depressed & 47 (37.6\%) & 22 (23.2\%) \\
\quad Never depressed & 78 (62.4\%) & 73 (76.8\%) \\
\addlinespace
\textbf{Child substance abuse status} & & \\
\quad Yes & 21 (16.8\%) & 7 (7.4\%) \\
\quad No & 104 (83.2\%) & 88 (92.6\%) \\
\addlinespace
\textbf{Social class of index parent} & & \\
\quad 1 (Highest) & 16 (12.8\%) & 1 (1.1\%) \\
\quad 2 & 19 (15.2\%) & 19 (20.0\%) \\
\quad 3 & 31 (24.8\%) & 16 (16.8\%) \\
\quad 4 & 50 (40.0\%) & 47 (49.5\%) \\
\quad 5 (Lowest) & 8 (6.4\%) & 12 (12.6\%) \\
\addlinespace
\textbf{Marital status of index parent} & & \\
\quad Married with spouse & 90 (72.0\%) & 84 (88.4\%) \\
\quad Separated/Divorced & 35 (28.0\%) & 11 (11.6\%) \\
\bottomrule
\end{tabular}
\end{table}

\textbf{Summary Statistics:}
\begin{itemize}
    \item Age at interview: Mean = 16.72 years (SD = 4.74, Range: 5--25)
    \item Age of depression onset (n=69): Mean = 15.19 years (SD = 4.74, Range: 4--25)
    \item Age of substance abuse onset (n=28): Mean = 21.46 years (SD = 3.53, Range: 12--25)
\end{itemize}

\newpage
\section{Descriptive Statistics Analyses}

\subsection{Sample Characteristics}
The study included 220 children (125 with depressed parents, 95 with non-depressed parents) aged 5--25 years (Mean = 16.72, SD = 4.74). Depression onset occurred in 69 children (31.4\%), with mean time to event of 15.19 years. Substance abuse was observed in 28 children (12.7\%), with mean time to event of 21.46 years. Among those who developed depression, 24 cases (34.8\%) had pre-pubertal onset (before age 13).

\subsection{Comparison by Parental Depression Status}

Table~\ref{tab:descriptive} shows demographic and clinical characteristics stratified by parental depression status. Children of depressed parents had higher rates of depression (37.6\% vs 23.2\%) and substance abuse (16.8\% vs 7.4\%). Separation/divorce was more common among depressed parents (28.0\% vs 11.6\%).

\vspace{1cm}

\noindent\textbf{Overall Kaplan-Meier Analysis:} Children of depressed parents demonstrated earlier depression onset (median survival: 20 vs.~23 years, log-rank $\chi^2$ = 7.69, p = 0.0056).

\begin{figure}[!ht]
    \centering
    \includegraphics[width=0.72\textwidth]{graph/overall_km_depression_by_parental_status.png}
    \caption{Overall Kaplan--Meier survival curves of depression onset by parental depression status}
    \label{fig:km_overall}
\end{figure}

\newpage
\section{Analyses to Address Hypothesis 1}

\textbf{Hypothesis 1:} Offspring of a depressed parent are more likely to have pre-pubertal onset ($<13$ years) depression than offspring of a non-depressed parent, but equally likely to have an onset of adolescent/early adulthood depression, given demographic and social characteristics.

\subsection{Statistical Modeling Approach}

Cox proportional hazards regression models were fit separately for: (a) pre-pubertal onset (age $<13$), and (b) adolescent/early adulthood onset (age $\geq 13$).

\noindent\textbf{Cox Model:}
\begin{equation}
h(t|X) = h_0(t) \times \exp(\beta_1 \times \text{PARDEP} + \beta_2 \times \text{SEX} + \beta_3 \times \text{AGE} + \beta_4 \times \text{SESCLASS} + \beta_5 \times \text{MSPARENT})
\end{equation}

Where: $h(t|X)$ = hazard at time $t$; $h_0(t)$ = baseline hazard; PARDEP = parental depression (1=depressed, 0=normal); SEX = child's sex (1=male, 2=female); AGE = age at interview; SESCLASS = social class (1--5); MSPARENT = marital status (1=married, 2=separated/divorced).

\noindent The proportional hazards (PH) assumption was assessed using the supremum test with 1000 replications, and results are reported for each hypothesis.
\newpage
\subsection{Results}

\subsubsection{Pre-pubertal Onset (Age $<13$ years)}

\textbf{Sample:} All 220 subjects included; 24 events (10.9\%), 196 censored at age 13.

\noindent\textbf{Kaplan-Meier (Figure~\ref{fig:km_prepub}):} Log-rank $\chi^2$ = 10.56, p = 0.0012. The survival curves show a clear separation, with children of depressed parents experiencing significantly earlier pre-pubertal depression onset compared to children of non-depressed parents.

\begin{figure}[H]
    \centering
    \includegraphics[width=0.7\textwidth]{graph/h1a_km_prepubertal_by_parental_status.png}
    \caption{H1a: Kaplan--Meier curves for pre-pubertal depression onset (age $<13$) by parental depression}
    \label{fig:km_prepub}
\end{figure}
\FloatBarrier

\begin{samepage}
\noindent\textbf{Cox Regression (Table~\ref{tab:cox_prepub}):}\nopagebreak
\begin{table}[H]
\centering
\caption{Cox Regression Results for Pre-pubertal Depression Onset (Age $<$ 13 years)}
\label{tab:cox_prepub}
\begin{tabular}{lccc}
\toprule
\textbf{Variable} & \textbf{Hazard Ratio} & \textbf{95\% CI} & \textbf{P-value} \\
\midrule
Parental Depression & 5.863 & (1.717, 20.019) & 0.0048** \\
Sex (Female vs Male) & 1.171 & (0.522, 2.626) & 0.7024 \\
Age at interview & 0.988 & (0.913, 1.069) & 0.7608 \\
Social class & 0.822 & (0.591, 1.144) & 0.2460 \\
Marital status (Sep/Div) & 0.585 & (0.198, 1.731) & 0.3328 \\
\bottomrule
\multicolumn{4}{l}{\small ** Statistically significant at $\alpha$ = 0.01 level}
\end{tabular}
\end{table}
\noindent\textbf{Key Findings:} Children of depressed parents have 5.86 times the hazard of pre-pubertal depression (HR=5.863, 95\% CI: 1.717--20.019, p=0.0048). No other covariates were significant. The wide confidence interval reflects the relatively small number of pre-pubertal depression events (n=24).

\noindent\textbf{Proportional Hazards Assessment:} The PH assumption was assessed using the supremum test with 1000 replications. All covariates satisfied the PH assumption: parental depression (p = 0.062), gender (p = 0.400), age (p = 0.184), social class (p = 0.090), and marital status (p = 0.228). The Cox model assumptions are met for all predictors in the pre-pubertal analysis.
\end{samepage}

\noindent\textit{Note: Complete SAS code for pre-pubertal analysis is provided in Appendix~A.2.}

\vspace{1cm}

\subsubsection{Adolescent/Early Adulthood Onset (Age $\geq$ 13 years)}

\textbf{Sample:} 154 subjects who reached age 13 without depression (excluded 66 subjects: 24 with pre-pubertal onset and 42 censored before age 13). 
% Note: 18 observations were excluded from Cox regression due to missing covariate data, resulting in 136 subjects with 38 events (27.9\%) and 98 censored (72.1\%) in the final analysis.

\noindent\textbf{Kaplan-Meier (Figure~\ref{fig:km_adol}):} Log-rank $\chi^2$ = 0.83, p = 0.3632 (not significant). No difference by parental status, consistent with hypothesis. 

% \noindent\textbf{Important Note on Time Variables:} The Kaplan-Meier curves display \textit{time since age 13} (i.e., years elapsed after entering the risk set at age 13) for visualization and interpretability. However, the Cox proportional hazards models use left-truncated time intervals $(\text{entry\_time}=13, \text{exit\_time}=\text{age at event/censoring})$ to properly account for delayed entry into the risk set. This approach ensures that individuals only contribute to the risk set after age 13, correctly handling the conditional nature of this analysis among those who survived depression-free through the pre-pubertal period.

\begin{figure}[H]
    \centering
    \includegraphics[width=0.7\textwidth]{graph/h1b_km_adolescent_by_parental_status.png}
    \caption{H1b: Kaplan--Meier curves for adolescent/early adulthood depression onset (age $\geq 13$) by parental depression}
    \label{fig:km_adol}
\end{figure}
\FloatBarrier

\begin{samepage}
\noindent\textbf{Cox Regression (Table~\ref{tab:cox_adol}):}\nopagebreak
\begin{table}[H]
\centering
\caption{Cox Regression Results for Adolescent/Early Adulthood Depression Onset (Age $\geq$ 13 years)}
\label{tab:cox_adol}
\begin{tabular}{lccc}
\toprule
\textbf{Variable} & \textbf{Hazard Ratio} & \textbf{95\% CI} & \textbf{P-value} \\
\midrule
Parental Depression & 1.100 & (0.567, 2.134) & 0.7779 \\
Sex (Female vs Male) & 2.226 & (1.125, 4.404) & 0.0216* \\
Age at interview & 0.979 & (0.847, 1.131) & 0.7695 \\
Social class & 1.139 & (0.828, 1.566) & 0.4240 \\
Marital status (Sep/Div) & 1.718 & (0.824, 3.583) & 0.1492 \\
\bottomrule
\multicolumn{4}{l}{\small * Statistically significant at $\alpha$ = 0.05 level}
\end{tabular}
\end{table}
\noindent\textbf{Key Findings:} Parental depression not significant (HR=1.100, 95\% CI: 0.567--2.134, p=0.778). Female sex predicts adolescent depression (HR=2.226, 95\% CI: 1.125--4.404, p=0.022), indicating females have more than double the hazard of adolescent-onset depression compared to males.

\noindent\textbf{Note on Sample Size:} The reduction from 154 to 136 subjects (18 excluded due to missing covariates) may affect power but does not alter the substantive conclusion that parental depression is not associated with adolescent-onset depression.

\noindent\textbf{Proportional Hazards Assessment:} Due to limited events (n=38), the Supremum test could not be computed. Exploratory analyses suggested potential PH violations for age (p=0.021). A sensitivity analysis using age$\times$log(time) interaction showed non-significant interaction (HR=2.194, p=0.257), with the main finding unchanged (parental depression HR=1.115, p=0.748).
\end{samepage}

\noindent\textit{Note: Complete SAS code for adolescent analysis with left-truncation is provided in Appendix~A.3.}

\vspace{1cm}


\subsection{Conclusion for Hypothesis 1}

\textbf{Hypothesis 1 is FULLY SUPPORTED.} Both components of the hypothesis were confirmed: 

(a)~Pre-pubertal analysis: Parental depression significantly increased depression risk (HR=5.863, 95\% CI: 1.717--20.019, p=0.0048), demonstrating that offspring of depressed parents are more likely to experience early-onset depression. 

(b)~Adolescent analysis: Parental depression showed no significant association (HR=1.100, 95\% CI: 0.567--2.134, p=0.778), confirming that offspring are equally likely to have adolescent-onset depression regardless of parental depression status. Female sex was the only significant predictor for adolescent-onset (HR=2.226, 95\% CI: 1.125--4.404, p=0.022). Time-interaction sensitivity analysis confirmed robustness of findings.

\newpage
\section{Analyses to Address Hypothesis 2}

\textbf{Hypothesis 2:} Is there evidence for the effect of (1) prior depression in offspring, as well as (2) the effect of parent's depression status on the age of onset of substance abuse/dependence in offspring, given demographic and social characteristics?

\subsection{Statistical Modeling Approach}

Three Cox regression models tested: (2a) prior depression effect, (2b) parental depression effect, (2c) joint model.

\noindent\textbf{Definition of ``Prior Depression'':} Prior depression is operationalized as having a history of depression during the observation period. For offspring who developed substance abuse, prior depression indicates depression onset preceded substance abuse onset (\texttt{BEDEPON} $<$ \texttt{BESUBON}). For those who did not develop substance abuse, prior depression indicates any history of depression by the interview age. This definition captures whether offspring with depression history have different substance abuse risk profiles, treating depression as a baseline characteristic rather than a time-varying covariate.

\textbf{Model Specifications:}
\begin{align}
\text{Model 2a: } h(t|X) &= h_0(t) \times \exp(\beta_1 \times \text{PRIORDEP} + \text{covariates}) \\
\text{Model 2b: } h(t|X) &= h_0(t) \times \exp(\beta_1 \times \text{PARDEP} + \text{covariates}) \\
\text{Model 2c: } h(t|X) &= h_0(t) \times \exp(\beta_1 \times \text{PRIORDEP} + \beta_2 \times \text{PARDEP} + \text{covariates})
\end{align}

Where PRIORDEP = prior depression (1=yes, 0=no); PARDEP = parental depression; covariates = sex, age, social class, marital status. The proportional hazards assumption was assessed for the joint model (Model 2c) using the same methods as Hypothesis 1.

\newpage
\subsection{Results}

\subsubsection{Model 2a: Effect of Prior Depression}

\textbf{Sample:} 220 subjects with complete data; 28 substance abuse events (12.7\%), 192 censored (87.3\%). 
% Note: One observation was excluded from regression analyses due to missing covariates, resulting in 219 subjects in the final models.

\noindent\textbf{Kaplan-Meier (Figure~\ref{fig:km_sub_prior}):} Log-rank $\chi^2$ = 0.07, p = 0.7988 (not significant). No difference in substance abuse onset by prior depression status.

\begin{figure}[H]
    \centering
    \includegraphics[width=0.7\textwidth]{graph/h2a_km_substance_onset_by_prior_depression.png}
    \caption{H2a: Kaplan--Meier curves for substance abuse onset by prior depression in offspring}
    \label{fig:km_sub_prior}
\end{figure}
\FloatBarrier

\begin{samepage}
\noindent\textbf{Cox Regression (Table~\ref{tab:cox_sub_prior}):}\nopagebreak
\begin{table}[H]
\centering
\caption{Cox Regression Results for Substance Abuse Onset: Effect of Prior Depression}
\label{tab:cox_sub_prior}
\begin{tabular}{lccc}
\toprule
\textbf{Variable} & \textbf{Hazard Ratio} & \textbf{95\% CI} & \textbf{P-value} \\
\midrule
Prior Depression & 0.999 & (0.454, 2.198) & 0.9981 \\
Sex (Female vs Male) & 0.538 & (0.247, 1.170) & 0.1155 \\
Age at interview & 1.045 & (0.889, 1.228) & 0.5981 \\
Social class & 1.000 & (0.692, 1.445) & 0.9989 \\
Marital status (Sep/Div) & 0.441 & (0.149, 1.303) & 0.1388 \\
\bottomrule
\end{tabular}
\end{table}
\noindent\textbf{Key Findings:} Prior depression shows no effect (HR=0.999, 95\% CI: 0.454--2.198, p=0.998). No covariates significant.
\end{samepage}

\vspace{1cm}

\subsubsection{Model 2b: Effect of Parental Depression}

\noindent\textbf{Kaplan-Meier (Figure~\ref{fig:km_sub_parent}):} Log-rank $\chi^2$ = 3.14, p = 0.0764 (marginally significant). Trend toward earlier substance abuse in children of depressed parents.

\begin{figure}[H]
    \centering
    \includegraphics[width=0.7\textwidth]{graph/h2b_km_substance_onset_by_parental_depression.png}
    \caption{H2b: Kaplan--Meier curves for substance abuse onset by parental depression}
    \label{fig:km_sub_parent}
\end{figure}
\FloatBarrier

\begin{samepage}
\noindent\textbf{Cox Regression (Table~\ref{tab:cox_sub_parent}):}\nopagebreak
\begin{table}[H]
\centering
\caption{Cox Regression Results for Substance Abuse Onset: Effect of Parental Depression}
\label{tab:cox_sub_parent}
\begin{tabular}{lccc}
\toprule
\textbf{Variable} & \textbf{Hazard Ratio} & \textbf{95\% CI} & \textbf{P-value} \\
\midrule
Parental Depression & 2.466 & (1.001, 6.076) & 0.0498* \\
Sex (Female vs Male) & 0.592 & (0.271, 1.293) & 0.1794 \\
Age at interview & 1.021 & (0.869, 1.200) & 0.8018 \\
Social class & 1.107 & (0.771, 1.591) & 0.5728 \\
Marital status (Sep/Div) & 0.378 & (0.128, 1.117) & 0.0801 \\
\bottomrule
\multicolumn{4}{l}{\small * Statistically significant at $\alpha$ = 0.05 level}
\end{tabular}
\end{table}
\noindent\textbf{Key Findings:} Parental depression marginally significant (HR=2.466, 95\% CI: 1.001--6.076, p=0.0498). Children of depressed parents have approximately 2.5 times the hazard of substance abuse.

\noindent\textbf{Proportional Hazards Assessment:} The PH assumption was assessed using the Supremum test with 1000 replications. All covariates satisfied the PH assumption with no significant violations detected (all p>0.05). The model assumptions are met for the parental depression analysis.
\end{samepage}

\vspace{1cm}

\subsubsection{Model 2c: Joint Model}

\begin{samepage}
\noindent\textbf{Cox Regression (Table~\ref{tab:cox_sub_joint}):}

\begin{table}[H]
\centering
\caption{Cox Regression Results for Substance Abuse Onset: Joint Model (Prior and Parental Depression)}
\label{tab:cox_sub_joint}
\begin{tabular}{lccc}
\toprule
\textbf{Variable} & \textbf{Hazard Ratio} & \textbf{95\% CI} & \textbf{P-value} \\
\midrule
Prior Depression & 0.819 & (0.372, 1.802) & 0.6273 \\
Parental Depression & 2.586 & (1.023, 6.539) & 0.0430* \\
Sex (Female vs Male) & 0.605 & (0.277, 1.322) & 0.2008 \\
Age at interview & 1.024 & (0.871, 1.203) & 0.7809 \\
Social class & 1.112 & (0.770, 1.606) & 0.5529 \\
Marital status (Sep/Div) & 0.366 & (0.121, 1.106) & 0.0740 \\
\bottomrule
\multicolumn{4}{l}{\small * Statistically significant at $\alpha$ = 0.05 level}
\end{tabular}
\end{table}

\noindent\textbf{Key Findings:} Parental depression remains significant (HR=2.586, 95\% CI: 1.023--6.539, p=0.043) in joint model. Prior depression not significant (HR=0.819, 95\% CI: 0.372--1.802, p=0.627). Parental depression is the primary driver of substance abuse risk, with children of depressed parents having approximately 2.6 times the hazard of substance abuse regardless of their own depression status.

\end{samepage}

\vspace{1cm}


\subsection{Conclusion for Hypothesis 2}

\textbf{Hypothesis 2 is PARTIALLY SUPPORTED.} Prior depression in offspring shows no association with substance abuse onset (Model 2a: HR=0.999, 95\% CI: 0.454--2.198, p=0.998). 

Parental depression demonstrates a marginally significant effect (Model 2b: HR=2.466, 95\% CI: 1.001--6.076, p=0.0498). 

The joint model (Model 2c) confirms that parental depression remains significant (HR=2.586, 95\% CI: 1.023--6.539, p=0.043) while prior depression does not (HR=0.819, p=0.627), indicating that parental depression directly affects substance abuse risk independent of offspring's own depression history. All models satisfied PH assumptions (all p>0.05).

\noindent\textit{Note: Complete SAS code for substance abuse analyses is provided in Appendix~A.4--A.7.}

% \newpage
% \section{Summary}

% \subsection{Main Findings}
% \begin{itemize}[leftmargin=1.2em, itemsep=0.35em, topsep=0.5em]
%     \item \textbf{Hypothesis 1 (FULLY SUPPORTED):} Parental depression strongly increases pre-pubertal depression risk (HR=5.86, $p<0.01$) but not adolescent-onset risk (HR=1.36, $p=0.32$). Age-specific familial transmission pattern identified.
%     \item \textbf{Hypothesis 2 (PARTIALLY SUPPORTED):} Prior depression shows no effect on substance abuse ($p=0.998$). Parental depression marginally significant (HR=2.47-2.59, $p=0.04$--$0.05$), operating independently of offspring depression.
%     \item \textbf{Additional Finding:} Female sex strongly predicts adolescent-onset depression (HR=2.48, $p<0.01$).
% \end{itemize}

% \subsection{Clinical Implications and Limitations}
% \vspace{0.3em}
% Early intervention critical for children of depressed parents, especially before age 13. Substance abuse prevention should target all children of depressed parents regardless of child's depression status.

% \textbf{Limitations:} Retrospective assessment may introduce recall bias; small event counts (n=24 pre-pubertal depression, n=28 substance abuse) limit precision and power; PH assumption violated for parental depression in adolescent analysis (addressed via stratified model).

\newpage
\appendix
\section{SAS Code for Statistical Analyses}
\label{appendix:code}

This appendix contains the complete SAS code used for all analyses reported in this study. The code is organized by analysis section corresponding to the main report.

\subsection{Data Import, Preparation, and Descriptive Statistics}

\begin{lstlisting}[style=sasstyle, caption={Data import and preparation}]
% 
/* Create analysis dataset with survival variables */
data survival_data;
    set rawdata;

    /* Depression survival variables */
    if DSMDEPHR = 0 and BEDEPON = -1 then do;
        censor = 1;              /* Censored (no depression) */
        time = PTAGE;            /* Use age at interview */
    end;
    else if DSMDEPHR = 1 and BEDEPON ne -1 then do;
        censor = 0;              /* Event occurred */
        time = BEDEPON;          /* Use age of onset */
    end;
    else if DSMDEPHR = 1 and BEDEPON = -1 then do;
        censor = 0;              /* Event occurred, onset unknown */
        time = PTAGE;            /* Use interview age */
    end;

    /* Rename variables for clarity */
    depress_parent = PARDEP;
    depress_child = DSMDEPHR;
    substance_abuse = DSMSUBHR;
    social_class = SESCLASS;
    marital_status = MSPARENT;
    age = PTAGE;
    gender = PTSEX;

    /* Create labeled categories */
    if gender = 1 then gender_label = "Male";
    else if gender = 2 then gender_label = "Female";

    if depress_parent = 0 then parent_label = "No Parental Depression";
    else if depress_parent = 1 then parent_label = "Parental Depression";

    /* Keep only valid observations */
    if time > 0 and time ne .;

    label
        censor = "Censoring Status (1=Censored, 0=Event)"
        time = "Time to Depression Onset or Censoring"
        depress_parent = "Parental Depression (0=No, 1=Yes)"
        social_class = "Social Class (1=Highest, 5=Lowest)"
        marital_status = "Marital Status (1=Married, 2=Sep/Div)"
        age = "Age at Interview"
        gender = "Gender (1=Male, 2=Female)";
run;
\end{lstlisting}

\begin{lstlisting}[style=sasstyle, caption={Descriptive statistics}]
/* Frequency tables by parental depression status */
proc freq data=survival_data;
    tables (gender depress_child substance_abuse social_class marital_status) 
           * parent_label / norow nocol nopercent;
    title "Descriptive Statistics by Parental Depression";
run;

/* Summary statistics for continuous variables */
proc means data=survival_data n mean std min max;
    var time age;
    title "Summary Statistics";
run;

/* Overall Kaplan-Meier curves for depression onset */
ods graphics on;
proc lifetest data=survival_data method=KM plots=survival(test);
    time time*censor(1);
    strata parent_label;
    title 'Overall KM Survival Curves: Depression Onset by Parental Status';
run;
ods graphics off;
\end{lstlisting}

\subsection{Hypothesis 1a: Pre-pubertal Depression Onset}

\begin{lstlisting}[style=sasstyle, caption={H1a: Pre-pubertal onset analysis}]
/* Create dataset for pre-pubertal analysis (<13 years) */
data prepubertal_data;
    set survival_data;
    
    if censor = 0 and time < 13 then do;
        time_prepub = time;      /* Event occurred before 13 */
        censor_prepub = 0;       /* Event */
    end;
    else do;
        time_prepub = 13;        /* Censored at age 13 */
        censor_prepub = 1;       /* Censored */
    end;
run;

/* Kaplan-Meier curves */
ods graphics on;
proc lifetest data=prepubertal_data method=KM plots=survival(test);
    time time_prepub*censor_prepub(1);
    strata parent_label;
    title 'H1a: Pre-pubertal Depression Onset (<13 years)';
run;
ods graphics off;

/* Cox regression */
proc phreg data=prepubertal_data;
    class depress_parent (ref='0') gender (ref='1') marital_status (ref='1');
    model time_prepub*censor_prepub(1) = depress_parent gender age 
          social_class marital_status / rl;
    title 'H1a: Cox Regression for Pre-pubertal Depression Onset';
run;

/* PH assumption test */
ods graphics on;
proc phreg data=prepubertal_data;
    class depress_parent (ref='0') gender (ref='1') marital_status (ref='1');
    model time_prepub*censor_prepub(1) = depress_parent gender age 
          social_class marital_status / ties=efron;
    assess PH / resample;
    title 'H1a: Proportional Hazards Assumption Test';
run;
ods graphics off;
\end{lstlisting}

\subsection{Hypothesis 1b: Adolescent Depression Onset}

\begin{lstlisting}[style=sasstyle, caption={H1b: Adolescent onset analysis with left truncation}]
/* Create dataset for adolescent analysis (>=13 years) */
data adolescent_data;
    set survival_data;
    
    /* Exclude those with depression before 13 or censored before 13 */
    if time < 13 then delete;
    
    /* Left truncation: all individuals enter at age 13 */
    entry_time = 13;
    exit_time = time;
    event = (censor = 0);
    
    /* For KM curves: time since age 13 */
    time_since_13 = exit_time - 13;
    censor_km = 1 - event;
run;

/* Kaplan-Meier curves */
ods graphics on;
proc lifetest data=adolescent_data method=KM plots=survival(test);
    time time_since_13*censor_km(1);
    strata parent_label;
    title 'H1b: Adolescent/Early Adulthood Depression Onset (>=13 years)';
run;
ods graphics off;

/* Cox regression with left truncation */
proc phreg data=adolescent_data;
    class depress_parent (ref='0') gender (ref='1') marital_status (ref='1');
    model (entry_time, exit_time)*event(0) = depress_parent gender age 
          social_class marital_status / rl;
    title 'H1b: Cox Regression for Adolescent/Early Adulthood Depression';
run;

/* Sensitivity analysis: age*time interaction */
proc phreg data=adolescent_data;
    class depress_parent (ref='0') gender (ref='1') marital_status (ref='1');
    model (entry_time, exit_time)*event(0) = depress_parent gender age 
          social_class marital_status age_time;
    age_time = age * log(exit_time);
    title 'H1b: Cox Model with Age*Time Interaction';
run;
\end{lstlisting}

\subsection{Hypothesis 2: Substance Abuse Dataset Preparation}

\begin{lstlisting}[style=sasstyle, caption={H2: Substance abuse analysis dataset preparation}]
/* Create substance abuse analysis dataset */
data substance_analysis;
    set rawdata;
    
    /* Substance abuse time and censoring */
    if DSMSUBHR = 1 and BESUBON ne -1 then do;
        time_subst = BESUBON;
        censor_subst = 0;
    end;
    else if DSMSUBHR = 1 and BESUBON = -1 then do;
        time_subst = PTAGE;
        censor_subst = 0;
    end;
    else if DSMSUBHR = 0 then do;
        time_subst = PTAGE;
        censor_subst = 1;
    end;
    
    /* Prior depression variable */
    prior_depression = 0;
    if DSMDEPHR = 1 and BEDEPON ne -1 then do;
        if DSMSUBHR = 1 and BESUBON ne -1 then do;
            if BEDEPON < BESUBON then prior_depression = 1;
        end;
        else if DSMSUBHR = 1 and BESUBON = -1 then prior_depression = 1;
        else if DSMSUBHR = 0 then prior_depression = 1;
    end;
    
    /* Rename variables */
    depress_parent = PARDEP;
    gender = PTSEX;
    age = PTAGE;
    social_class = SESCLASS;
    marital_status = MSPARENT;
    
    if time_subst > 0 and time_subst ne .;
run;
\end{lstlisting}

\subsection{Hypothesis 2a: Effect of Prior Depression}

\begin{lstlisting}[style=sasstyle, caption={H2a: Effect of prior depression}]
/* Kaplan-Meier curves */
ods graphics on;
proc lifetest data=substance_analysis method=KM plots=survival(test);
    time time_subst*censor_subst(1);
    strata prior_depression;
    title 'H2a: Substance Abuse Onset by Prior Depression in Offspring';
run;
ods graphics off;

/* Cox regression */
proc phreg data=substance_analysis;
    class prior_depression (ref='0') gender (ref='1') marital_status (ref='1');
    model time_subst*censor_subst(1) = prior_depression gender age 
          social_class marital_status;
    title 'H2a: Cox Regression - Effect of Prior Depression';
run;
\end{lstlisting}

\subsection{Hypothesis 2b: Effect of Parental Depression}

\begin{lstlisting}[style=sasstyle, caption={H2b: Effect of parental depression}]
/* Kaplan-Meier curves */
ods graphics on;
proc lifetest data=substance_analysis method=KM plots=survival(test);
    time time_subst*censor_subst(1);
    strata depress_parent;
    title 'H2b: Substance Abuse Onset by Parental Depression Status';
run;
ods graphics off;

/* Cox regression */
proc phreg data=substance_analysis;
    class depress_parent (ref='0') gender (ref='1') marital_status (ref='1');
    model time_subst*censor_subst(1) = depress_parent gender age 
          social_class marital_status;
    title 'H2b: Cox Regression - Effect of Parental Depression';
run;

/* PH assumption test */
ods graphics on;
proc phreg data=substance_analysis;
    class depress_parent (ref='0') gender (ref='1') marital_status (ref='1');
    model time_subst*censor_subst(1) = depress_parent gender age 
          social_class marital_status / ties=efron;
    assess PH / resample;
    title 'H2b: Proportional Hazards Assumption Test';
run;
ods graphics off;
\end{lstlisting}

\subsection{Hypothesis 2c: Joint Model}

\begin{lstlisting}[style=sasstyle, caption={H2c: Joint model}]
/* Joint model with both prior and parental depression */
proc phreg data=substance_analysis;
    class prior_depression (ref='0') depress_parent (ref='0') 
          gender (ref='1') marital_status (ref='1');
    model time_subst*censor_subst(1) = prior_depression depress_parent 
          gender age social_class marital_status;
    title 'H2c: Cox Regression - Joint Model (Prior and Parental Depression)';
run;
\end{lstlisting}

\end{document}
